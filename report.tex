\documentclass[12pt]{article}

\title{Final Project: CPE 367: Touch Tone Filter}
\author{Brian Mere, Thomas Choboter}
\date{\today}

% lec_pre.sty  

%────────────────────────────────────────────────────────────────────────────────────────────────────────────────────────────────────────────────────
% The following is part of the needed stuff for inkscape integration ... 
\usepackage{import}
\usepackage{xifthen}
\usepackage{pdfpages}
\usepackage{transparent}

\newcommand{\incfig}[1]{%
    \def\svgwidth{\columnwidth}
    \import{./figures/}{#1.pdf_tex}
}
%────────────────────────────────────────────────────────────────────────────────────────────────────────────────────────────────────────────────────
\usepackage[utf8]{inputenc}
\usepackage[a4paper, total={6in, 8in}]{geometry}
\usepackage[english]{babel}
\usepackage[]{amsthm} %lets us use \begin{proof}
\usepackage[]{amssymb} %gives us the character \varnothing
\usepackage{fancyhdr}
\usepackage{amsmath}
\usepackage[hidelinks]{hyperref}
\usepackage{svg}
\usepackage{float}
\usepackage{wasysym} % Used for smileys
\usepackage{fancyhdr}
\usepackage{pgfplots}
\usepackage{verbatim}
\usepackage{minted}
\newtheorem{definition}{Definition} % Definition ... definition
\newtheorem{example}{Example}
\newtheorem{theorem}{Theorem}
\newtheorem{post}{Postulate}
\newtheorem{corollary}{Corollary}
\newtheorem{proposition}{Proposition}
\newtheorem{postulate}{Postulate}
\newtheorem{lemma}{Lemma}
\newtheorem{proof_method}{Proof Method}
\newtheorem{axiom}{Axiom}

\makeatletter
\let\mytitle\@title
\let\myauthor\@author
\let\mydate\@date
\makeatother

\newcommand{\myclass}{CPE 367}
\newcommand{\myclassname}{Digital Signals and Systems}

\pgfplotsset{compat=1.18}
\setlength{\headheight}{15pt}
\setlength{\parindent}{0pt}

\pagestyle{fancy}
\fancyhf{}
\lhead{\mytitle}
\rhead{\myauthor, \\ \mydate}
\renewcommand{\headrulewidth}{1pt}

% Math operator commands %

\DeclareMathOperator\card{\text{card}}

\newcommand{\bb}[1]{\mathbb{#1}}
\newcommand{\cl}[1]{\mathcal{#1}}
\newcommand{\Laplace}[1]{\cl{L}\left\{#1\right\}}
\newcommand{\ILaplace}[1]{\cl{L}^{-1}\left\{#1\right\}}
\newcommand{\lprod}[1]{\left\langle #1\right\rangle}
\newcommand{\Vc}[1]{\bold{#1}}

\DeclareMathOperator{\proj}{proj}
\DeclareMathOperator{\spane}{span}


%%%%%%%%%%%%%%%%%%%%%%%%%
%% TITLE PAGE DEFS ------
%%%%%%%%%%%%%%%%%%%%%%%%%

\usepackage[english]{babel}
\usepackage{inputenc}
\usepackage{graphicx}
\usepackage[colorinlistoftodos]{todonotes}


\newcommand{\inserttitlepage}[0]{

    \begin{titlepage}

    \newcommand{\HRule}{\rule{\linewidth}{0.5mm}} % Defines a new command for the horizontal lines, change thickness here

    \center % Center everything on the page
    
    %----------------------------------------------------------------------------------------
    %	HEADING SECTIONS
    %----------------------------------------------------------------------------------------

    \textsc{\LARGE California Polytechnic State University}\\[2.0cm] % Name of your university/college
    \textsc{\Large \myclass}\\[0.5cm] % Major heading such as course name
    \textsc{\large \myclassname}\\[0.5cm] % Minor heading such as course title

    %----------------------------------------------------------------------------------------
    %	TITLE SECTION
    %----------------------------------------------------------------------------------------

    \HRule \\[0.4cm]
    { \huge \bfseries \mytitle}\\[0.4cm] % Title of your document
    \HRule \\[1.5cm]

    %----------------------------------------------------------------------------------------
    %	AUTHOR SECTION
    %----------------------------------------------------------------------------------------

    \begin{minipage}{0.4\textwidth}
        \begin{flushleft} \large
        \emph{Author:}\\
        \myauthor\\ % Your name 
        \emph{Date:}\\
        {\large \today}\\[2cm] % Date, change the \today to a set date if you want to be precise
        \end{flushleft}
        \end{minipage}

    \hfill

    \begin{figure}[b]
        \centerline{
            \includegraphics[width=\textwidth/2]{
                C:/Users/brian/Documents/SchoolPapers/Repositories/CPE367Lab/LabReports/Template/CP_logo_alt_cmyk.jpg
            }% Include a department/university logo - this will require the graphicx package
        }
    \end{figure}
    

    %----------------------------------------------------------------------------------------

    \vfill % Fill the rest of the page with whitespace

    \end{titlepage}
}
 

\newcommand{\magphaseplot}[1]{
    \begin{figure}[H]
        \centering 
        \includesvg[width=0.6\textwidth]{media/#1_resp.svg}
        \caption{$f = #1$ Bandpass Filter Response}
        \label{fig:f=#1}
    \end{figure}
}

\begin{document} 

\inserttitlepage 

\section*{Abstract}

We used a common Goertzel Algorithmic design to create a touch-tone 
analyzer, to predict what tone is being pressed within a given 
signal of a variety of frequencies. We detail our design below:

\section*{Results}

Overall, we got an ISI error of \texttt{16.1\%} on the fastest signal, and 
got an even better \texttt{6.5\%} error on the slower signal. The following 

For our filter coefficients, we used a $b = 20$ value for our 
bits as a good guideline. Increasing the number of bits did very 
little past this point, as it only increased the resolution of 
our data points, and not necessarily the correctness of them. 

What follows is our runs on both \texttt{dtmf\_signals\_slow.txt}
(see Figure \ref{fig:slow})
and \texttt{dtmf\_signals\_fast.txt} 
(see Figure \ref{fig:fast})
each:

\begin{figure}
    \centering 
    \includesvg{media/slow.svg}
    \caption{Output from \texttt{dtmf\_signals\_slow.txt}}
    \label{fig:slow}
\end{figure}

\begin{figure}
    \centering 
    \includesvg{media/fast.svg}
    \caption{Output from \texttt{dtmf\_signals\_fast.txt}}
    \label{fig:fast}
\end{figure}


\section*{High-Level Design}

Our filters use an algorithm known as \textbf{Goertzel's Algorithm}.
It essentially boils down to:

\begin{enumerate}
    \item An IIR Bandpass filter centered at some frequency $\omega_0$. 
    \item An FIR filter to extract the DFT from the incoming data.
\end{enumerate}

The IIR Bandpass filter is described as follows via $s[n]$:

\[
    s[n] = x[n] + 2\cos(\omega_0)s[n-1] - s[n-2]
\]

Notice that this is just our $H(z)$ Bandpass filter, 
with values of $\omega_0$ for the center 
frequency, as well as using $r = 1$.

The FIR filter converts our $s[n]$ into $y[n]$ so that 
we can get the DFT at our $\omega_0$ by calculating $y[N]$ for 
some input signal of length $N$. It's defined as:

\[
    y[n] = s[n] - e^{-j \omega_0}s[n-1]
\]

The math behind how this caluclates the DFT is as follows. Notice 
that we can find the $Z$-transforms for both $S$ and $Y$ as follows:

\begin{align*}
    \frac{S(z)}{X(z)} &= \frac{1}{1 - 2\cos(\omega_0)z^{-1} + z^{-2}} \\
    &= \frac{1}{(1-e^{j\omega_0}z^{-1})(1-e^{-j\omega_0}z^{-1})}
\end{align*}

\begin{align*}
    \frac{Y(z)}{S(z)} &= 1 - e^{-j\omega_0}z^{-1}
\end{align*}

Combining the two filters gives:

\begin{align*}
    \frac{S(z)}{X(z)}\frac{Y(z)}{S(z)} &= \frac{Y(z)}{X(z)} \\
    &= H(z) \\
    &= \frac{1 - e^{-j\omega_0}z^{-1}}{(1-e^{j\omega_0}z^{-1})(1-e^{-j\omega_0}z^{-1})} \\
    &= \frac{1}{1 - e^{+j\omega_0}z^{-1}}
\end{align*}

Which can be transformed back into: 

\begin{align*}
    y[n] &= x[n] + e^{+j\omega_0}y[n-1] \\
    &= \sum_{k = -\infty}^n x[k]e^{+j\omega_0(n-k)}
\end{align*}

We impose that when $k < 0$ that $x[k] = 0$, which is usually 
the case, so then:

\begin{align*}
    y[n] &= \sum_{k = 0}^n x[k]e^{+j\omega_0(n-k)} \\
    &= e^{+j\omega_0} \sum_{k=0}^n x[k]e^{-j\omega_0k}
\end{align*}

Where the right side is just the DFT! The power of this 
algorithm is that we want to just calculate $y[N]$, but notice that
we can substitute the definition of $s[n]$ to get:

\begin{align*}
    y[N] &= s[N] - e^{-j\omega_0}s[N-1] \\
    &= (2\cos(\omega_0)s[N-1] - s[N-2]) - e^{-j\omega_0}s[N-1] \\
    &= e^{+j\omega_0}s[N-1] - s[N-2] 
\end{align*}

Thus, the algorithm essentially boils down to:

\begin{enumerate}
    \item Calculate, using $x[n]$ all $s[0] \to s[N-1]$ terms using the definition of $s[n]$. 
    \item Use the shortcut described above to get $y[N]$
    \item Return $|y[N]|$ as your DFT magnitude for $\omega_0$.
\end{enumerate}

We do note that $\omega_0 = 2\pi \frac{k}{N}$, so the frequency 
you want to measure must satisfy that $k$ is an integer in the equation.
Usually, even if some $f$ we want to measure isn't an integer 
multiple, we take the nearest $k$ as a good approximation. 

As such, a high-level diagram of our design is as follows:

\begin{figure}[H]
    \centering 
    \includesvg[width=1.0\textwidth]{media/Touch Tone High Level Diagram.svg}
    \caption{High Level Diagram}
    \label{fig:high-level}
\end{figure}

We define each ``group'' of frequencies we want to analyze and 
choose a ``best'' values as a \texttt{GoertzelComb}. Here, in 
this case the frequencies \texttt{697, 770, 852, 941} form a 
\texttt{GoertzelComb}.

As such, as long as \texttt{GoertzelFilter} works, we get 
a best guess for each group of frequencies, giving us the tone 
that we detect. 

\section*{Low-Level Design}

Implementing the difference equations above, we get the 
following low-level diagram for our design:

\begin{figure}[H]
    \centering 
    \includesvg[width=1.0\textwidth]{media/Touch Tone Low Level Diagram.svg}
    \caption{Low Level Diagram}
    \label{fig:low-level}
\end{figure}

Our \texttt{DSP} for using integer arithmetic is as follows:

\begin{figure}[H]
    \centering 
    \includesvg[width=1.0\textwidth]{media/Touch Tone DSP.svg}
    \caption{DSP Diagram}
    \label{fig:dsp}
\end{figure}

\section*{Coefficients Used}

We note that we generalized a lot of the process of finding 
the coefficients used for most difference equations. As such, the 
equations are generated ``on-the-fly'' based on the input 
frequencies we need. The good thing is that all frequencies are 
constant, so we can generate a list of the constants used. As 
such, we used the following \texttt{float} constants, and used 
the \texttt{DSP} above to get the following coefficients:

\begin{minted}{text}
    
    Coefficients for ALL filters: (Using r = 20, N = 32)
    --------------------------------------------
    b0 = 1048576 -> b0(float) = 1.0
    a1 = 1048576 -> a0(float) = 1.0

    Coefficients for each filter by frequency:
    ==========================================
    
    Low-Frequencies:
    ----------------
    f = 697Hz  -> a0 = -802545, a0(float) = -0.7653668647301797
    f = 770Hz  -> a0 = -802545, a0(float) = -0.7653668647301797
    f = 852Hz  -> a0 = -409134, a0(float) = -0.39018064403225666
    f = 941Hz  -> a0 = 0      , a0(float) = -1.2246467991473532e-16

    High-Frequencies:
    ----------------
    f = 1209Hz -> a0 = 802545 , a0(float) = 0.7653668647301795
    f = 1336Hz -> a0 = 1165115, a0(float) = 1.111140466039204
    f = 1447Hz -> a0 = 1482910, a0(float) = 1.414213562373095
    f = 1633Hz -> a0 = 1743718, a0(float) = 1.6629392246050907
\end{minted}

These come from the following \texttt{matlab} file to generate 
the coefficients and the following magnitude-phase plots:

\inputminted{matlab}{calc.m}

Which gives the following plots:

\section*{Codebase}

We used a lot of files for this project, and don't want to 
fill up the contents of this report. Please check out 
\href{https://github.com/BrianMere/CPE367FinalProject.git}{\textbf{this link (click
on me)}}
for a full list of current files used in our design (use the 
\texttt{main} branch).

\magphaseplot{697}
\magphaseplot{770}
\magphaseplot{852}
\magphaseplot{941}

\magphaseplot{1209}
\magphaseplot{1336}
\magphaseplot{1477}
\magphaseplot{1633}

Notice the following:

\begin{itemize}
    \item The \textit{true} center frequencies of the bandpass filters 
    are \textit{not} the values of $f_m$ we pass in. Rather, due to the 
    nature of the DFT, we have to use an integer multiple of $2\pi/N$, so 
    the frequency must get rounded to the nearest value of $\omega_0$ such that 
    $k$ is indeed an integer.
    \item The impulse repsonse itself is periodic, so it should 
    \textit{never} reach a stable state of decreasing by 50\% or less 
    by some unit of time.  
\end{itemize}




\section*{Additional Discussion}

Some things of interest that were covered during this project:

\begin{enumerate}
    \item We had to really nail down using a correct Goertzel Algorithm. 
    Many approaches \textit{actually} don't use complex arithmetic 
    for their calculations, but convert to separate real and 
    imaginary numbers for the values of question.
    \item We tried to before use a BP filter \textit{before} sending it 
    to \texttt{Goertzel}; however, it's better not to overfilter 
    the signal as otherwise the input is very sensitive to deviations 
    away from the target frequency. As such, we used the fact that 
    \texttt{Goertzel} itself contains a BP filter as a justification of 
    not using an external one into it. 
    \item For outputting no-symbol, we hypothesized that we could 
    try to output a no-symbol signal \textit{if} all of the $X[k]$ for 
    each frequency in question is about the same. If that's the case, it's 
    most likely that there is no symbol, as the frequencies we are detecting 
    are likely just noise. 
    \item It's good that we used the DTMF standard values for frequencies to 
    detect. This is because they're pretty relatively spread out, so 
    the values of $k$ for each frequency is, at the very least, unique (for some 
    big enough $N$).
    \item Some ways to improve this design includes trying to use different $N$ 
    values for the rows and columns of frequencies. We could use a smaller $N$ 
    value for higher frequencies (when trying to choose $k$), so to have 
    a more up-to-date way of updating the high and low frequencies, we could 
    offset these $N$ values to have one update at faster times. 
\end{enumerate}
    







\end{document}